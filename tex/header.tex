% Patch für LaTeX laden:
%\RequirePackage{fixltx2e}

%------------------------------------------------------------------------------
%---------------- Docmentenklasse, Layout und Ränder: -------------------------
%------------------------------------------------------------------------------

\documentclass[
        paper=a4,               % Papierformat DIN A4
        fontsize=12pt,
        BCOR=12mm,              % 12mm Binderandkorrektur
        parskip=half,           % Absätze als halbe Leerzeile
        headsepline,            % Linie zwischen Kopfzeile und Dokument
        cleardoublepage=plain,  % Keine Kopf/Fußzeile auf Leerseiten
        bibliography=totoc,     % Bibliographie als nicht-nummeriertes 
                                % Kapitel im Inhaltsverzeichnis
        open=right,             % Kapitel beginnen immer aufrechten Seiten
        %open=any,               % Kapitel dürfen auf beiden Seiten beginnen
        numbers=noenddot,       % Keine Punkte nach Abbildungsnummern
        captions=tableheading,  % Spacing für Captions über Tabellen an
        titlepage=firstiscover, % Titelseite ist Deckblatt, symmetrische Ränder
        headings=normal,         % kleinere Überschriften
        %draft,
    ]{scrbook}


% Warnung, falls erneut kompilirt werden muss:
\usepackage[aux]{rerunfilecheck}

% Beschränkung auf chapter und section im Inhaltsverzeichnis:            
\setcounter{tocdepth}{2}

%------------------------------------------------------------------------------
%------------------------------ Sprache und Schrift: --------------------------
%------------------------------------------------------------------------------

% Deutsche Spracheinstellungen
%\usepackage{polyglossia}
%\setdefaultlanguage{german}
\usepackage[ngerman]{babel}

% stellt den \enquote{} Befehl

% verbessert das allgemeine Schriftbild:
\usepackage{microtype}

\usepackage{fontspec}
\defaultfontfeatures{Ligatures=TeX}


\usepackage{xcolor}

% Astroparticle shortcuts
\usepackage{hepnicenames}
\usepackage{isotope}


%------------------------------------------------------------------------------
%------------------------ Für die Matheumgebung--------------------------------
%------------------------------------------------------------------------------

\usepackage{amsmath}
\usepackage{amssymb}
\usepackage{mathtools}

% Enable Unicode-Math and follow the ISO-Standards for typesetting math
\usepackage[
  math-style=ISO,
  bold-style=ISO,
  sans-style=italic,
  nabla=upright,
  partial=upright,
]{unicode-math}
\setmathfont{Latin Modern Math}
\usepackage[autostyle=true,german=quotes]{csquotes}

% schöne Brüche im Text mit \sfrac{}{}
\usepackage{xfrac}  


%Gleichungsnummern Kapitel.Unterkapitel.Gleichung
\renewcommand{\theequation}{\thesection{}.\arabic{equation}}
\numberwithin{equation}{chapter}
\numberwithin{equation}{section}

\newcommand{\ab}{\\\\}
\newcommand{\qa}[1]{\textbf{#1}\mbox{}\\}
\newcommand{\absatz}{\\\\}
\newenvironment{qanda}{\setlength{\parindent}{0pt}}{\bigskip}
\newcommand{\Q}{\bigskip\bfseries}
\newcommand{\A}{\par\textbf{} \normalfont}

%------------------------------------------------------------------------------
%---------------------------- Einheiten ---------------------------------------
%------------------------------------------------------------------------------

\usepackage[
  locale=DE,
  separate-uncertainty=true,
  per-mode=symbol-or-fraction,
]{siunitx}
\sisetup{math-micro=\text{µ},text-micro=µ}

%------------------------------------------------------------------------------
%------------------------------ Tabellen: -------------------------------------
%------------------------------------------------------------------------------

\usepackage{booktabs}       % stellt \toprule, \midrule, \bottomrule
\usepackage{threeparttable} % für komplexere Tabellen

%------------------------------------------------------------------------------
%------------------------------ Grafiken: -------------------------------------
%------------------------------------------------------------------------------

\usepackage[]{graphicx} % enhances includegraphics
\usepackage{grffile}  % adds support for a larger range of file names

\usepackage{scrhack}
\usepackage{float}
\floatplacement{figure}{htbp}
\floatplacement{table}{htbp}

\usepackage{caption}
\captionsetup{%
            labelfont=bf,               % Label fett
            format=plain,               % Caption-Text steht auch unter "Tabelle x"
            width=0.9\textwidth,        % Bereich für Caption schmaler als für Text
            font=small,
           }

\usepackage{subcaption}   % for subfigures
%\usepackage{rotating}
%\usepackage[above,below,section]{placeins} % forces floating objects to stay within a given border
%\usepackage{flafter}  % floats only appear after they are referenced


%------------------------------------------------------------------------------
%------------------------------ Bibliographie ---------------------------------
%------------------------------------------------------------------------------

\usepackage[%
    backend=biber,
    style=alphabetic,]{biblatex}    % Biblatex mit biber
    \addbibresource{content/references.bib}     % die Bibliographie einbinden
\DeclareBibliographyAlias{misc}{article}
\DefineBibliographyStrings{german}{andothers = {{et\,al\adddot}}}


%------------------------------------------------------------------------------
%------------------------------ Sonstiges: ------------------------------------
%------------------------------------------------------------------------------



\usepackage[pdfusetitle,unicode]{hyperref}
\usepackage{bookmark}
\usepackage[shortcuts]{extdash}
\usepackage{bbding} % used for \XSolidBrush x
\usepackage{pdflscape}
% forbid typographic malpractice
\clubpenalty = 10000
\widowpenalty = 10000
\displaywidowpenalty = 10000

\renewcaptionname{ngerman}\sectionautorefname{Abschnitt}
\renewcaptionname{ngerman}\chapterautorefname{Kapitel}
\renewcaptionname{ngerman}\subsectionautorefname{Unterabschnitt}
\renewcaptionname{ngerman}\figureautorefname{Abbildung}