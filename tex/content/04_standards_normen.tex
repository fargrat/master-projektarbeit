\chapter{Regionale Zusicherung des ethischen Umgangs mit künstlicher Intelligenz}\label{chap:standards}
Im Folgenden werden Mittel zur Zusicherung ethischen Umgangs mit Anwendungen der künstlichen Intelligenz beispielhaft auf internationaler Ebene, sowie für die Regionen Europa, USA und Asien betrachtet.
Die Zusicherung erfolgt aktuell meist durch Standards und Normen, wobei spezielle Vorgaben für Reinforcement Learning zum Zeitpunkt der Recherche nicht auffindbar sind.
Deshalb wird die Betrachtung auf das Obergebiet der künstlichen Intelligenz ausgeweitet, da die meisten Vorgaben so weit gefasst sind, dass sie ihre Gültigkeit auch für den Teilbereich des Reinforcement Learning behalten.
Eine Betrachtung dieser regionalen Zusicherung ist aus mehreren Gründen interessant.
Die frühzeitige Beachtung der Standards ist insbesondere für die spätere technische Konzeption sinnvoll, da die Vorgaben möglicherweise in der Zukunft als Vorlage für Gesetze dienen können.
Eine Auseinandersetzung mit Standards und Normen signalisiert Anwendern ein Interesse des Herausgebers an ethischen Fragestellungen.
Die Werte sind zudem abhängig von den jeweiligen Rechts- und Kulturräumen.
Die Vorgaben können die dahinterliegenden Wertmaßstäbe, sowie den Stellenwert der Thematik im jeweiligen Raum aufzeigen.

\section{International}\label{sec:international}
Auf internationaler Ebene werden Standards durch die International Organization for Standardization \cite{isoabout} (kurz ISO) entwickelt.
Die ISO besteht aus aus 164 länderspezifisischen Standardisierungseinrichtungen, welche unabhängig von den einzelnen Regierungen sind.
So ist beispielsweise in Deutschland die jeweilige Standardisierungseinrichtung das Deutsche Institut für Normung \cite{din} (kurz DIN), in der USA das American National Standards Institute \cite{ansi} (kurz ANSI) und in China die Standardization Administration China \cite{sac} (kurz SAC).
Ziel ist die Konsolidierung lokaler Standards und damit die Schaffung von Standards auf internationaler Ebene, um möglichst alle Dienste und Produkte \enquote{safe, reliable and of good quality} \cite{iso_benefits} zu gestalten.
Um das zu erreichen sind die Mitglieder in technische Komitees unterteilt, welche die tatsächlichen Standards entwickeln.
Jedem Komitee wird ein Themenbereich zugewiesen, welcher dann Standards entwickelt, diskutiert und später aktualisiert.
\ab 
Da der Prozess der Etablierung von Standards langwierig ist, werden explizit Normen betrachtet, die auf die Oberthematik der Ethik in IT-Systemen allgemein bezogen sind.
Bestehende Normen, wie die ISO-Norm 13482:2014 für private Roboter und die ISO 10218:2012 \cite[S. 50]{wittpahl2019} für industriell eingesetzte Roboter geben Normen für einen Anwendungsbereich des Reinforcement Learning vor, gehen allerdings nicht auf die Technologie explizit ein.
Eine speziellere Betrachtung der Thematik des maschinellen Lernens erfolgt durch das Subkomitee ISO/IEC JTC 1/SC 42.
Zu den hier vorgeschlagenen Normen gehören \cite{iso_sc42}:
\begin{itemize}
    \item \textbf{ISO/IEC CD 22989} \cite{iso22989}: Definition grundlegender Konzepte und Begriffe.
    \item \textbf{ISO/IEC CD 23053} \cite{iso23053}: Definition eines Frameworks zur Nutzung von maschinellen Lernens.
    \item \textbf{ISO/IEC AWI 23894} \cite{iso23894}: Risikomanagement für Künstliche Intelligenz.
    \item \textbf{ISO/IEC AWI TR 24368} \cite{iso24368}: Ethische und soziale Bedenken.
\end{itemize}
Die meisten der Normen des Subkomitees sind allerdings zum aktuellen Zeitpunkt erst in der Konzeptionsphase und müssen noch diskutiert, geprüft und veröffentlicht werden, bevor sie in Kraft treten.
Insgesamt liefert die ISO einen wichtigen Beitrag zur Erstellung weltweit gemeinsamer Standards.
Die Normen sind nicht rechtsbindend, werden aber teilweise als Grundlage für Gesetze genutzt und bieten die Möglichkeit für Herausgeber entsprechender Systemen die Einhaltung dieser Standards nachzuweisen.
\section{Europa}\label{sec:europa}
In Europa gibt es neben nationalen Vereinigungen, wie Bitkom \cite{bitkom} oder der Plattform Lernende Systeme \cite{pls} auch übernationale Vereinigungen, wie Claire \cite{claire}, AI4People \cite{ai4people} oder der HEG-KI \cite{smuha}.
Die Vereinigungen verstehen sich zum Teil als Dachverbände und Vertreter teilnehmender Unternehmen oder als Gruppen, welche sich aus Experten der Forschung und Wirtschaft zusammensetzen.
Sie unterscheiden sich durch die Zusammensetzung der Mitglieder, sowie durch den Aufbau und die Ziele.
Im Fall von AI4People bestehen die Mitglieder z.B. aus internationalen Unternehmen, wie Facebook, Intel und Microsoft.
Im Gegensatz dazu besteht beispielsweise die Plattform Lernende Systeme etwa zum gleichen Anteil aus Vertretern der Wirtschaft und der Wissenschaft und ist in sieben Arbeitsgruppen zu unterschiedlichen Domänen des Oberthemas lernende Systeme aufgeteilt.
Dort gibt es z.B. explizit eine Arbeitsgruppe zum Thema Ethik von lernenden Systemen \cite{plsa}.

\subsection*{Ethik-Leitlinien der HEG-KI}
Im Folgenden sollen exemplarisch die Ethik-Leitlinien der Hochrangingen Expertengruppe für Künstliche Intelligenz (kurz HEG-KI) betrachtet werden.
Die Expertengruppe wird offiziell von der europäischen Kommission eingesetzt und ist an alle Beteiligten gerichtet.
Also vom Bürger über den Entwickler bis zu Unternehmen und Behörden.
Es wird Wert auf den europäischen Ursprung der entwickelten Inhalte gelegt, um Unabhängigkeit zu wahren.
Inhaltliche Grundlage ist die Verbesserung der Lebensqualität der Bürger, die Umsetzung von Nachhaltigkeit und die Minimierung möglicher Risiken.
Das Handeln soll gemäß der europäischen Werte \enquote{Menschenrechte, Demokratie und Rechtsstaatlichkeit} \cite[S. 6]{smuha} erfolgen.
Die Pfeiler dieser Werte sind laut der HEG-KI Rechtmäßigkeit, Ethik und Robustheit.
Ziel ist die Definition einer vertrauenswürdigen künstlichen Intelligenz.
Dafür werden Grundsätze und Bewertungskriterien zur \enquote{Entwicklung, Einführung und Nutzung von KI-Systemen} \cite[S. 3]{smuha} aufgezeigt.
Das Vorgehen soll den gesamten Prozess der Entwicklung begleiten, möglichst unabhängig sein und sieht einen dauerhaften Abgleich mit den Anforderungen vor.
Es ist eingeteilt in Fundamente, Verwirklichung und Bewertung.
Die Zusicherung und Umsetzung der Maßnahmen ist unterteilt in technische- und nichttechnische Maßnahmen, wodurch möglichst alle Beteiligten miteinbezogen werden soll.
\ab 
Ziel des Dokumentes ist die Standardisierung für die Entwicklung und Nutzung ethischer KI-Anwendungen.
Es soll Vertrauen innerhalb der Gesellschaft schaffen, indem grundlegende Konzepte erklärt und jegliche Entscheidungen fachlich begründet werden.
Allen Beteiligten soll ein Vorgehen aufgezeigt werden, mit dem ethische KI-Anwendungen strukturiert werden können.
Auch wenn die Leitlinien zunächst nicht bindend sind, ist die Definition sinnvoll.
So haben Herausgeber durch die Richtlinien frühzeitig die Möglichkeit, dem Gesetzgeber, sowie Anwendern ein Bemühen bei der Umsetzung ethischer KI-Anwendungen nachzuweisen.
\section{USA}\label{sec:usa}
In der USA gibt es das IEEE \cite{ieee} als weltweit größter Fachverband mit über 420\,000 Mitgliedern aus technischen Berufen in 160 Ländern \cite[S. 287]{chatila2019}.
Ziel ist die Förderung von Innovation und Technologie zum Wohle des Menschen zu lenken.
Deshalb hat das IEEE in \citetitle{chatila2019} diverse Aspekte des ethischen Softwaredesign betrachtet.
\ab 
Neben dem allgemeinen Vorgehen werden Grundprinzipien, rechtliche Grundlagen, sowie Tipps zur Implementierung aufgelistet.
Zu den Grundprinzipien zählen z.B. die Einhaltung und Beachtung der Menschenrechte, sowie die physische und emotionale Unversehrtheit des Menschen, die Nachvollziehbarkeit, Effektivität und ein hinreichendes Maß an Kompetenz der Entwickler.
Inhaltlich werden explizit Entwickler angesprochen.
Ziel des Dokumentes ist die Verbesserung des menschlichen Lebens durch KI-gestützte Maschinen, die explizit ethische Richtlinien befolgen und so den Menschen dienen \cite[S. 6]{chatila2019}.
Die allgemeine Beschreibung stellt dabei den Anfang einer Reihe von tatsächlichen Standards in der IEEE-Gruppe \enquote{P7000 - Model Process for Addressing Ethical Concerns During System Design}\cite[S. 283]{chatila2019} \cite{emelc-wg} dar.
\section{Asien}\label{sec:asien}
Durch ein großes Engagement asiatischer Länder im Bereich der Förderung und Forschung \cite[S. 46]{tsinghuauniversity} Künstlicher Intelligenz werden im Folgenden Maßnahmen asiatischer Länder betrachtet.
So haben exemplarisch dafür im Jahr 2017 23 \% aller KI-Unternehmen ihren Sitz in China gehabt \cite[S. 142]{ding}.
China selbst hat mehrere Pläne veröffentlicht \cite[S. 8]{ding}, um das Vorgehen für die Zukunft in unterschiedlichen Teilbereichen der künstlichen Intelligenz zu fördern, um bis 2030 das weltweite Zentrum der Forschung und Anwendung von KI zu sein.
Deshalb wird im Folgenden betrachtet, wie ethische Kriterien asiatischer Länder zugesichert werden soll.
\ab 
China, Singapur, Australien, Malaysia und Indien haben bereits Institutionen gebildet \cite{mit2019}, um ethische KI-Fragen zu diskutieren.
So hat Indien beispielsweise mit \#AIforAll \cite{nitiaayog2018} die nationale Strategie veröffentlicht, in der insbesondere Fokus auf den Nutzen für die Gesellschaft und ethischen Umgang gesetzt wird.
Lediglich Japan, Südkorea und Taiwan legen aktuell wenig bis keine Priorität auf ethische Bedenken von KI Anwendungen \cite{mit2019}, auch wenn einige dieser Länder in der Vergangenheit schon nicht rechtlich verbindliche Standards diskutiert haben.
Viele Universitäten engagieren sich dabei für internationale Lösungen und arbeiten eng mit ihren jeweiligen Regierungen zusammen.
Ziel ist der Aufbau von Vertrauen in die bestehenden Menschen- und Datenschutzrechte, ohne dabei Fortschritt und Innovation zu behindern.
Nahezu alle Länder sind an der Diskussion beteiligt und sehen Potenzial in der Technologie als zukünftigen Wirtschaftszweig.
Dennoch werden beispielsweise von China Anwendungen, wie die flächendeckende Nutzung von Gesichtserkennung \cite{mozur2018} genutzt, welche in der Form nicht vereinbar mit unseren europäischen Werten sind.