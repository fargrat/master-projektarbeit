\chapter{Zusammenfassung}\label{chap:zusammenfassung}
Im Rahmen dieser Arbeit wurden technische und organisatorische Maßnahmen zur Zusicherung ethischer Werte in Reinforcement-Learning-Anwendungen aufgezeigt.
Zur Umsetzung dieser Maßnahmen wurde dann ein Vorgehensplan entwickelt, der mit üblichen Softwareentwicklungsabläufen kompatibel ist.
Als Grundlage für die Definition der ethischen Werte und der Konzeption des Vorgehensplans wurden zunächst Grundlagen der angewandten Ethik und Moral vorgestellt.
Dies beeinhaltet eine Abgrenzung der Begriffe Ethik und Moral, die Entstehung von Normen und Werten, sowie die Einordnung der Begriffe in einen praktischen Kontext.
Innerhalb der Ethik lassen sich Situationen und Anwendungen je nach Kontext in Bereichsethiken einteilen.
So wurden Grundlagen der Maschinenethik als Bereichsethik des Reinforcement Learning eingeführt und diskutiert, inwiefern Software-Agenten die Möglichkeiten besitzen, moralisch zu handeln.
Neben den ethischen Grundlagen wurden technische Grundlagen des Reinforcement Learning betrachtet.
Hierbei wurden explizit Markov-Entscheidungsprozess als Grundlage der Abbildung von Reinforcement-Learning-Umgebungen eingeführt, sowie allgemeine Eigenschaften und Begriffe der Technologie aufgezeigt.
Als Basis für die Definition der ethischen Werte, die ein Reinforcement-Learning-Agent erfüllen soll, um moralisch handeln zu können, wurden anschließend die Maßnahmen und Forderungen von Europa, USA und Asien, sowie verschiedenen Institutionen, wie der Hochrangingen Expertengruppe für künstliche Intelligenz in Europa, IEEE in der USA und der ISO auf internationaler Ebene analysiert.
Anschließend wurde der Kontext, sowie Technologien und Vorgehen innerhalb der Arbeit abgegrenzt und Anwendungsgebiete des Reinforcement Learning aufgezeigt, sowie inhaltlich verwandte Arbeiten verglichen.
Als spätere Bewertungsgrundlagen wurden dann Herausforderungen aufgeführt, die bei der Definition der ethischen Werte, sowie der Konzeption der Maßnahmen zur Zusicherung eben dieser relevant sind.
In der eigentlichen Konzeption wurden dann die ethischen Werte Nachvollziehbarkeit und Erklärbarkeit, Vertrauen durch Kalkulierbarkeit und Zuverlässigkeit, sowie Verantwortung und Schuld im einzelnen definiert und die Relevanz im Kontext der Arbeit begründet.
Zur Zusicherung der ethischen Werte wurde anschließend ein Vorgehensplan konzipiert, welcher möglichst verfahrensunabhängig entlang eines generischen Softwareentwicklungsablaufs technische organisatorische Maßnahmen zu den verschiedenen Entwicklungsphasen Vorbereitung, Konzeption und Umsetzung, Test und Wartung aufzeigt.
Die Maßnahmen umfassen Möglichkeiten zur Selektion geeigneter Verfahren, Sensibilisierung der Beteiligten, Nachvollziehbarkeit von Datenverarbeitung, Prozessen und der Entscheidungen des Agenten, sowie der sicheren Erkundung in fremden Umgebungen.
Ebenso werden Maßnahmen zur Zusicherungen der Korrektheit und der nachvollziehbaren Evaluierung von Agenten, sowie zur transparenten Wartung der nachhaltigen Zusicherung von Vertrauen aufgezeigt.
Abschließend wurde der im Rahmen der Arbeit erarbeitete Vorgehensplan anhand der anfänglichen Fragestellungen und Herausforderungen bewertet und Möglichkeiten zur Weiterentwicklungen betrachtet.

