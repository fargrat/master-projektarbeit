\section{USA}\label{sec:usa}
In der USA gibt es das IEEE \cite{ieee} als weltweit größter Fachverband mit über 420\,000 Mitgliedern aus technischen Berufen in 160 Ländern \cite[S. 287]{chatila2019}.
Ziel ist die Förderung von Innovation und Technologie zum Wohle des Menschen zu lenken.
Deshalb hat das IEEE in \citetitle{chatila2019} diverse Aspekte des ethischen Softwaredesign betrachtet.
\ab 
Neben dem allgemeinen Vorgehen werden Grundprinzipien, rechtliche Grundlagen, sowie Tipps zur Implementierung aufgelistet.
Zu den Grundprinzipien zählen z.B. die Einhaltung und Beachtung der Menschenrechte, sowie die physische und emotionale Unversehrtheit des Menschen, die Nachvollziehbarkeit, Effektivität und ein hinreichendes Maß an Kompetenz der Entwickler.
Inhaltlich werden explizit Entwickler angesprochen.
Ziel des Dokumentes ist die Verbesserung des menschlichen Lebens durch KI-gestützte Maschinen, die explizit ethische Richtlinien befolgen und so den Menschen dienen \cite[S. 6]{chatila2019}.
Die allgemeine Beschreibung stellt dabei den Anfang einer Reihe von tatsächlichen Standards in der IEEE-Gruppe \enquote{P7000 - Model Process for Addressing Ethical Concerns During System Design}\cite[S. 283]{chatila2019} \cite{emelc-wg} dar.