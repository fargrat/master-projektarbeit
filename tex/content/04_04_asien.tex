\section{Asien}\label{sec:asien}
Durch ein großes Engagement asiatischer Länder im Bereich der Förderung und Forschung \cite[S. 46]{tsinghuauniversity} Künstlicher Intelligenz werden im Folgenden Maßnahmen asiatischer Länder betrachtet.
So haben exemplarisch dafür im Jahr 2017 23 \% aller KI-Unternehmen ihren Sitz in China gehabt \cite[S. 142]{ding}.
China selbst hat mehrere Pläne veröffentlicht \cite[S. 8]{ding}, um das Vorgehen für die Zukunft in unterschiedlichen Teilbereichen der künstlichen Intelligenz zu fördern, um bis 2030 das weltweite Zentrum der Forschung und Anwendung von KI zu sein.
Deshalb wird im Folgenden betrachtet, wie ethische Kriterien asiatischer Länder zugesichert werden soll.
\ab 
China, Singapur, Australien, Malaysia und Indien haben bereits Institutionen gebildet \cite{mit2019}, um ethische KI-Fragen zu diskutieren.
So hat Indien beispielsweise mit \#AIforAll \cite{nitiaayog2018} die nationale Strategie veröffentlicht, in der insbesondere Fokus auf den Nutzen für die Gesellschaft und ethischen Umgang gesetzt wird.
Lediglich Japan, Südkorea und Taiwan legen aktuell wenig bis keine Priorität auf ethische Bedenken von KI Anwendungen \cite{mit2019}, auch wenn einige dieser Länder in der Vergangenheit schon nicht rechtlich verbindliche Standards diskutiert haben.
Viele Universitäten engagieren sich dabei für internationale Lösungen und arbeiten eng mit ihren jeweiligen Regierungen zusammen.
Ziel ist der Aufbau von Vertrauen in die bestehenden Menschen- und Datenschutzrechte, ohne dabei Fortschritt und Innovation zu behindern.
Nahezu alle Länder sind an der Diskussion beteiligt und sehen Potenzial in der Technologie als zukünftigen Wirtschaftszweig.
Dennoch werden beispielsweise von China Anwendungen, wie die flächendeckende Nutzung von Gesichtserkennung \cite{mozur2018} genutzt, welche in der Form nicht vereinbar mit unseren europäischen Werten sind.