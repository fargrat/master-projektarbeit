\subsection{Vorbereitungsphase}\label{sub:vorbereitungsphase}
In der Vorbereitsphase sollen zunächst grundlegende Informationen gesammelt und der Anwendungskontext klar abgegrenzt werden.
Dafür werden zusätzlich organisatorische Anforderungen aufgezeigt, um die ethischen Werte zu berücksichtigen.
Es soll eine klare Definition des Einsatzzweckes, der Absichten und der Anforderungen definiert werden und so eine solide Grundlage für die späteren Phasen entstehen.

\subsubsection{Sensibilisierung der Beteiligten}
Zur Umsetzung der Maßnahmen müssen alle beteiligten Personen für das Thema Ethik sensibilisiert sein.
Dadurch können bereits frühzeitig personelle Probleme erkannt und Lösungen gefunden werden.
Die Motivation für die Durchsetzung ethischer Maßnahmen kann sowohl von außen, also extrinsisch, als auch von den Personen selber, also intrinsisch, erfolgen \cite{baum2017}.
Intrinsische Motivation kann nur begrenzt beeinflusst werden und ist hauptsächlich vom Wertesystem des Einzelnen, aber auch von seiner Umgebung abhängig.
Einfluss kann darauf durch extrinsische Maßnahmen ausgeübt werden, indem beispielsweise die sozialen Normen innerhalb des Unternehmens Ethik als Kernthema enthalten und durch unabhängige Instanzen überprüft und eingefordert werden.
Um Teil der Gemeinschaft zu sein kann sich so beim Einzelnen eine intrinsische Motivation entwickeln, die mit den Unternehmensnormen übereinstimmt.
Auf der anderen Seiten kann eine extrinsische Motivation durch das Unternehmen oder durch die Politik und Gesellschaft eingefordert werden.
Maßnahmen dazu sind beispielsweise Gesetze, welche den Einsatz bestimmter Technologien verbieten.
Ebenso können Vorgaben des Unternehmens und daraus potenziell resultierende Strafen oder umgekehrt Belohnungen bei Einhaltung eine Möglichkeit sein, um ethisch korrektes Handeln zu motivieren.
\subsubsection{Definition des Anwendungskontext}
Im Folgenden werden Maßnahmen und Fragestellungen betrachtet, die zur klaren Definition des Einsatzzweckes, der Absichten und Anforderungen relevant sind.
Die Maßnahmen sollten bereits bei der Erhebung der Anforderungen beachtet werden.
\\\\
\\\\
\begin{qanda}
    \Q Welche Aufgaben soll das System haben? 
    \A Eine realistische und präzise Definition der Aufgaben und Ziele des Systeme bietet die Grundlage der Entwicklung.
    Gemäß \enquote{AI is not magic} \cite[S. 13]{gottesman2018} sind hier bereits Schwächen und Grenzen der zu nutzenden Technologie zu beachten.
    
    \Q In welchem Kontext soll das System eingesetzt werden?
    \A Zu beachten sind Rechts- und Kulturräume, öffentliche und private Einsatzzwecke, sowie Einzel- und Multiagentenumgebungen und die Einbettung in andere IT-Systeme.
    
    \Q Welche Limitierungen soll das System haben und welchen Werten soll es folgen?
    \A KI-gestütze Systeme sollten einen klaren Anwendungskontext besitzen und dementsprechend auch Limitierungen ihrer Funktionalität.
    Die klare Definition und öffentliche Kommunikation dieser Limitierungen kann helfen, Vertrauen aufzubauen.
    Ebenso sollte frühzeitig entschieden werden, welchen Werten das System folgen soll, da die Abbildung der Werte in gesamten Produktlebenszyklus beachtet werden muss.
    
    \Q Soll der Agent explizit oder implizit ethisch handeln?
    \A Implizit ethische Agenten sind durch ihren Anwendungskontext nicht in der Lage unethisch zu handeln. 
    In dem Fall muss ein Fokus darauf gelegt werden, den Agenten auf genau diesen Kontext zu begrenzen.
    Ist dies nicht der Fall, müssen für den Agenten explizit Maßnahmen ergriffen werden, um die ethischen Werte umzusetzen.
    Diese Maßnahmen werden in den Folgenden Phasen vorgestellt.
    
    \Q Wie erfolgt der Einfluss auf die Umgebung, insbesondere auf Menschen?
    \A Agenten können z.B. als Expertensystemen einen indirekte Einfluss auf ihre Umwelt besitzen, indem die Entscheidungen von anderen Systemen ausgeführt werden müssen.
    Im Gegensatz dazu besitzen Agenten mit direktem Einfluss auf die Umwelt Aktoren, um selbständig mit der Umwelt zu interagieren.
    Die Entscheidung, um welche Art von Agent es sich im Bezug auf den Einfluss auf die Umwelt handelt, ist essentiell für die spätere Beachtung der Maßnahmen und hat einen großen Einfluss auf die Kritikalitätsbewertung.
\end{qanda}

Als Ergebnis entsteht neben einer notwendigen Grundlage für den späteren Entwicklungsprozess eine Kritikalitätsbewertung.
Die Kritikalitätsbewertung gibt Aussage darüber, welche Mitarbeiter am Projekt beteiligt sein dürfen, wie der Umgang mit den dazugehörigen Daten aussehen muss, welche Verfahren zu wählen sind, was maximale Eingriffs- und Anpassungszeiten im Fehlerfall sind und in welchem Maße die Prozesse geprüft und zertifiziert werden müssen.
Die in den späteren Phasen beschriebenen Maßnahmen können eine Hilfestellung geben diese Fragen zu beantworten.

