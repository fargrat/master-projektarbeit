\section{International}\label{sec:international}
Auf internationaler Ebene werden Standards durch die International Organization for Standardization \cite{isoabout} (kurz ISO) entwickelt.
Die ISO besteht aus aus 164 länderspezifisischen Standardisierungseinrichtungen, welche unabhängig von den einzelnen Regierungen sind.
So ist beispielsweise in Deutschland die jeweilige Standardisierungseinrichtung das Deutsche Institut für Normung \cite{din} (kurz DIN), in der USA das American National Standards Institute \cite{ansi} (kurz ANSI) und in China die Standardization Administration China \cite{sac} (kurz SAC).
Ziel ist die Konsolidierung lokaler Standards und damit die Schaffung von Standards auf internationaler Ebene, um möglichst alle Dienste und Produkte \enquote{safe, reliable and of good quality} \cite{iso_benefits} zu gestalten.
Um das zu erreichen sind die Mitglieder in technische Komitees unterteilt, welche die tatsächlichen Standards entwickeln.
Jedem Komitee wird ein Themenbereich zugewiesen, welcher dann Standards entwickelt, diskutiert und später aktualisiert.
\ab 
Da der Prozess der Etablierung von Standards langwierig ist, werden explizit Normen betrachtet, die auf die Oberthematik der Ethik in IT-Systemen allgemein bezogen sind.
Bestehende Normen, wie die ISO-Norm 13482:2014 für private Roboter und die ISO 10218:2012 \cite[S. 50]{wittpahl2019} für industriell eingesetzte Roboter geben Normen für einen Anwendungsbereich des Reinforcement Learning vor, gehen allerdings nicht auf die Technologie explizit ein.
Eine speziellere Betrachtung der Thematik des maschinellen Lernens erfolgt durch das Subkomitee ISO/IEC JTC 1/SC 42.
Zu den hier vorgeschlagenen Normen gehören \cite{iso_sc42}:
\begin{itemize}
    \item \textbf{ISO/IEC CD 22989} \cite{iso22989}: Definition grundlegender Konzepte und Begriffe.
    \item \textbf{ISO/IEC CD 23053} \cite{iso23053}: Definition eines Frameworks zur Nutzung von maschinellen Lernens.
    \item \textbf{ISO/IEC AWI 23894} \cite{iso23894}: Risikomanagement für Künstliche Intelligenz.
    \item \textbf{ISO/IEC AWI TR 24368} \cite{iso24368}: Ethische und soziale Bedenken.
\end{itemize}
Die meisten der Normen des Subkomitees sind allerdings zum aktuellen Zeitpunkt erst in der Konzeptionsphase und müssen noch diskutiert, geprüft und veröffentlicht werden, bevor sie in Kraft treten.
Insgesamt liefert die ISO einen wichtigen Beitrag zur Erstellung weltweit gemeinsamer Standards.
Die Normen sind nicht rechtsbindend, werden aber teilweise als Grundlage für Gesetze genutzt und bieten die Möglichkeit für Herausgeber entsprechender Systemen die Einhaltung dieser Standards nachzuweisen.