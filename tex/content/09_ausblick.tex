\chapter{Ausblick}\label{chap:ausblick}
Insbesondere im Hinblick auf die Adaption von Reinforcement-Learning-Anwendungen der breiten Masse bietet diese Arbeit Potenzial für zukünftige Weiterentwicklungen.
Dazu gehören Maßnahmen zur Schaffung von praktischen Nutzungskontexten und die Kommunikation eben dieser, der Förderung von Bildungs- und Informationsmöglichkeiten und der Reduzierung der Nutzungskomplexität.
\ab
Ebenso sollte die praktische Umsetzung der Maßnahmen, insbesondere der Dokumentation stärker vorgegeben werden, um eine nachvollziehbare Erfüllung der Maßnahmen nachweisen zu können.
Maßnahmen dazu wären beispielsweise der Nachweis der Erfüllung der Maßnahmen und der Dokumentation in einer standardisierten aber erweiterbaren Form mit der Anwendung auszuliefern.
Insbesondere um die Akzeptanz aus technischer Sicht zuzusichern wäre eine exemplarische Beispielanwendung gemäß des vorgestellten Vorgehens sinnvoll.
\ab 
Auch wenn das Vorgehen möglichst verfahrensunabhängig ist, so ist Reinforcement Learning als Technologie schnelllebig.
In Zukunft ist deshalb eine regelmäßige Evaluation der technischen Maßnahmen im Hinblick auf die Aktualität der Maßnahmen und der Nutzung neuer Verfahren, sowie der Erweiterung der organisatorischen Maßnahmen notwendig.