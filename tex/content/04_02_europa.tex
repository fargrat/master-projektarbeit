\section{Europa}\label{sec:europa}
In Europa gibt es neben nationalen Vereinigungen, wie Bitkom \cite{bitkom} oder der Plattform Lernende Systeme \cite{pls} auch übernationale Vereinigungen, wie Claire \cite{claire}, AI4People \cite{ai4people} oder der HEG-KI \cite{smuha}.
Die Vereinigungen verstehen sich zum Teil als Dachverbände und Vertreter teilnehmender Unternehmen oder als Gruppen, welche sich aus Experten der Forschung und Wirtschaft zusammensetzen.
Sie unterscheiden sich durch die Zusammensetzung der Mitglieder, sowie durch den Aufbau und die Ziele.
Im Fall von AI4People bestehen die Mitglieder z.B. aus internationalen Unternehmen, wie Facebook, Intel und Microsoft.
Im Gegensatz dazu besteht beispielsweise die Plattform Lernende Systeme etwa zum gleichen Anteil aus Vertretern der Wirtschaft und der Wissenschaft und ist in sieben Arbeitsgruppen zu unterschiedlichen Domänen des Oberthemas lernende Systeme aufgeteilt.
Dort gibt es z.B. explizit eine Arbeitsgruppe zum Thema Ethik von lernenden Systemen \cite{plsa}.

\subsection*{Ethik-Leitlinien der HEG-KI}
Im Folgenden sollen exemplarisch die Ethik-Leitlinien der Hochrangingen Expertengruppe für Künstliche Intelligenz (kurz HEG-KI) betrachtet werden.
Die Expertengruppe wird offiziell von der europäischen Kommission eingesetzt und ist an alle Beteiligten gerichtet.
Also vom Bürger über den Entwickler bis zu Unternehmen und Behörden.
Es wird Wert auf den europäischen Ursprung der entwickelten Inhalte gelegt, um Unabhängigkeit zu wahren.
Inhaltliche Grundlage ist die Verbesserung der Lebensqualität der Bürger, die Umsetzung von Nachhaltigkeit und die Minimierung möglicher Risiken.
Das Handeln soll gemäß der europäischen Werte \enquote{Menschenrechte, Demokratie und Rechtsstaatlichkeit} \cite[S. 6]{smuha} erfolgen.
Die Pfeiler dieser Werte sind laut der HEG-KI Rechtmäßigkeit, Ethik und Robustheit.
Ziel ist die Definition einer vertrauenswürdigen künstlichen Intelligenz.
Dafür werden Grundsätze und Bewertungskriterien zur \enquote{Entwicklung, Einführung und Nutzung von KI-Systemen} \cite[S. 3]{smuha} aufgezeigt.
Das Vorgehen soll den gesamten Prozess der Entwicklung begleiten, möglichst unabhängig sein und sieht einen dauerhaften Abgleich mit den Anforderungen vor.
Es ist eingeteilt in Fundamente, Verwirklichung und Bewertung.
Die Zusicherung und Umsetzung der Maßnahmen ist unterteilt in technische- und nichttechnische Maßnahmen, wodurch möglichst alle Beteiligten miteinbezogen werden soll.
\ab 
Ziel des Dokumentes ist die Standardisierung für die Entwicklung und Nutzung ethischer KI-Anwendungen.
Es soll Vertrauen innerhalb der Gesellschaft schaffen, indem grundlegende Konzepte erklärt und jegliche Entscheidungen fachlich begründet werden.
Allen Beteiligten soll ein Vorgehen aufgezeigt werden, mit dem ethische KI-Anwendungen strukturiert werden können.
Auch wenn die Leitlinien zunächst nicht bindend sind, ist die Definition sinnvoll.
So haben Herausgeber durch die Richtlinien frühzeitig die Möglichkeit, dem Gesetzgeber, sowie Anwendern ein Bemühen bei der Umsetzung ethischer KI-Anwendungen nachzuweisen.